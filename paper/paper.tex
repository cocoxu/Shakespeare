%% Example of a LaTeX source file for a COLING-2012 submission
%% last updated: July 10, 2012
%% Optional instructions for authors within the tex file are provided as comments and start with 'for authors:...'
\documentclass[10pt,a5paper,twoside]{article}
\usepackage{coling2012}
\title{Translating to Shakespeare: A Case Study in Paraphrasing Writing Styles}
%for authors: in case of more than four author names ref. to commented line below 
%\author{$Annie~SMITH^{1, 2}~~~LI~Xiao Dong^{1, 3}$\\$~~~Third~Author^{1, 2}~~~Fourth~Author^{1, 3}~~~ Fifth~Author^{2, 3}$\\
\author{$Author1^{1, 2}~~~Author2^{1, 3}$\\
{\small  	(1) INSTITUTE\_1, address 1\\ 
 		(2) INSTITUTE\_2, address 2\\
		(3) INSTITUTE\_3, address 3\\
  \texttt{author1@institute1, author@institute2} \\ 
}}

\begin{document}
\maketitle
%% The first mandatory ABSTRACT (\abstractEn) section below is for the English language
\abstractEn{  %ABSTRACT}{
We present innitial investigation into the task of paraphrasing writing style.
The plays of William Shakespeare and their \emph{modern} translations are treated
as parallel text which is used to learn paraphrase models targeting the style of Early Modern English employed by Shakespeare.
We demonstrate that existing evaluation metrics developed in the Machine Translation and Paraphrase communities are
innappropriate when generating paraphrases targeting a specific writing style, and
propose a new metric to measure how closely the generated paraphrases match the target
style.  To the best of our knowledge no previous work has investigated the task of
paraphrasing text with the goal of targeting a specific style of writing.
}

\keywordsEn{Paraphrase, Writing Style}

% section 2
\section{Introduction}
% subsection 2.1

\section{Data}

\section{Evaluation Metrics}

\section{Experiments}

\section{Conclusions}

\bibliographystyle{apa}

\bibliography{colingbiblio}
\nocite{TALN2007,LaigneletRioult09,LanglaisPatry07,au1972,cks1981,mb2012}

%%================================================================
\end{document}
